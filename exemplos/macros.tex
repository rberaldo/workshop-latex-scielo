%
% macros.tex
%
% Workshop de LaTeX do SciELO
%
% Demonstra:
% - Declaração de macros de substituição
% - Uso do pacote xspace
% - Redefinição de macros e customização do \maketitle
% - Criação de macros que levam argumentos
%

\documentclass[a4paper]{article}
\usepackage{polyglossia}
  \setdefaultlanguage{brazil}
  \setotherlanguages{english}
\usepackage{blindtext}
\usepackage{xcolor}
\usepackage{xspace}

% Macro de substituição
\newcommand{\scielo}{SciELO\xspace}

% Macro com argumento
\newcommand{\eng}[1]{%
  \emph{\textenglish{#1}}%
}

\title{Macros: como automatizar funções no \LaTeX}
\author{Rafael Beraldo}
\date{14 de junho de 2017}

% Redefinindo o comando \maketitle
\makeatletter % Para acessar comandos como \@author
\renewcommand{\maketitle}{
  \begin{flushleft}
    % Todo o ambiente flushleft terá fonte sem serifa
    \sffamily
    % Vamos colocar uma régua com a largura da linha e a espessura de 1pt
    {\rule{\linewidth}{1pt}}
    % Título
    {\Large\bfseries\@title\par}
    \medskip
    % Autor
    {\color{darkgray}\large\itshape por \@author{}\par
    % Data
    dia \@date \par}
    % Mais uma régua
    {\rule{\linewidth}{1pt}}
    % Espaçamento entre o título e resto do documento
    \bigskip
  \end{flushleft}
}
\makeatother

\begin{document}
\frenchspacing

\maketitle

Uma das maiores vantagens do \LaTeX{} em relação aos outros editores de texto é
a sua extensibilidade. É possível adicionar funcionalidades ao sistema por meio
de \emph{macros}. O próprio \LaTeX{} não passa de um conjunto (bastante
complexo) de macros do \TeX.

Essas macros são programas que automatizam certas funções e permitem que o
autor foque em escrever, ao invés de realizar tarefas tediosas repetidamente.

\section{Macros de substituição}

O tipo mais simples de macro é o de substituição: \scielo. (Ver definição
no preâmbulo do documento.)

\section{Macros com variáveis}

Em \LaTeX, é comum usar um~\verb+@+ para escrever macros que não devem ser
facilmente acessíveis pelo usuário, como é o caso de \verb+\@author+. Vejamos o
exemplo no preâmbulo de como utilizar esses comandos para redefinir o título.

\section{Macros com argumentos}

Este é uma parágrafo que contém \eng{some sentences in English which are
hyphenated according to the correct rules}, se necessário.

\end{document}
