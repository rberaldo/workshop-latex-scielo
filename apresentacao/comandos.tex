% Comandos %%%%%%%%%%%%%%
% Sintaxe dos comandos LaTeX
\begin{frame}[standout]
  \Huge
  Comandos
\end{frame}

% Comandos simples não têm argumentos
\begin{frame}[fragile]
  \frametitle{Comandos simples}
  \begin{minted}[autogobble,fontsize=\huge,breaklines]{latex}
    \tableofcontents
  \end{minted}
\end{frame}

% Comandos simples e espaço em branco
\begin{frame}[fragile]
  \frametitle{Comandos simples e espaço em branco}
  \begin{minted}[autogobble,fontsize=\normalsize,breaklines]{latex}
    \tableofcontents Embora isso funcione, o próximo exemplo ganha mais pontos por estilo e ajuda na leitura do código. Por quê?
  \end{minted}
\end{frame}

\begin{frame}[fragile]
  \frametitle{Comandos simples e espaço em branco}
  \begin{minted}[autogobble,fontsize=\normalsize,breaklines]{latex}
    \tableofcontents
    Esse exemplo é melhor, mas como espaço em branco não faz diferença, talvez valesse a pena colocar mais uma linha entre o parágrafo e o comando.
  \end{minted}
\end{frame}

% Comandos com argumentos
\begin{frame}[fragile]
  \frametitle{Comandos com argumento}
  \begin{minted}[autogobble,fontsize=\normalsize,breaklines]{latex}
    \section{Introdução}\label{introducao}Este exemplo funciona, mas o código não é muito legível. O resultado será perfeito, entretanto.
  \end{minted}
\end{frame}

% Voltemos ao exemplo para demonstrar
\begin{frame}[standout]
  \Huge
  \filename{exemplo/artigo.tex}
\end{frame}
