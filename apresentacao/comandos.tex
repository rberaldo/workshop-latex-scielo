% Comandos %%%%%%%%%%%%%%
% Sintaxe dos comandos LaTeX
\begin{frame}[standout]
  \Huge
  Comandos
\end{frame}

% Comandos simples não têm argumentos
\begin{frame}[fragile]
  \frametitle{Comandos simples}
  \begin{minted}[autogobble,fontsize=\huge,breaklines]{latex}
    \tableofcontents
  \end{minted}
\end{frame}

% Comandos simples e espaço em branco
\begin{frame}[fragile]
  \frametitle{Comandos simples e espaço em branco}
  \begin{minted}[autogobble,fontsize=\Large,breaklines]{latex}
    \tableofcontents Isso funciona
  \end{minted}
\end{frame}

\begin{frame}[fragile]
  \frametitle{Comandos simples e espaço em branco}
  \begin{minted}[autogobble,fontsize=\Large,breaklines]{latex}
    \tableofcontents
    Melhor agora
  \end{minted}
\end{frame}

% Comandos com argumentos
\begin{frame}[fragile]
  \frametitle{Comandos com argumento}
  \begin{minted}[autogobble,fontsize=\normalsize,breaklines]{latex}
    \section{Introdução}\label{introducao}Também funciona
  \end{minted}
\end{frame}

% Voltemos ao exemplo para demonstrar
\begin{frame}[standout]
  \Huge
  \filename{exemplo/artigo.tex}
\end{frame}
